% Copyright 2018 Google LLC
%
% Use of this source code is governed by an MIT-style
% license that can be found in the LICENSE file or at
% https://opensource.org/licenses/MIT.

%!TeX spellcheck = en-US

\documentclass[eprint.tex]{subfiles}
\begin{document}
\section{Security reduction}
HBSH is a tweakable, variable-input-length, secure pseudorandom permutation: an attacker
succeeds if they distinguish it from a family of independent random permutations indexed by
input length and tweak, given access to both encryption and decryption oracles.

$K_E$ and $K_H$ are derived from $K_S$:
$K_E || K_H || \ldots = S_{K_S}(\lambda)$. HBSH is then
the conjugation of an inner transform by an outer:
\begin{align*}
    \HBSH:
    \bin^* &\times \bin^l \times \bin^n \rightarrow \bin^l \times \bin^n \\
    \HBSH_{T} &= \phi^{-1}_{K_H, T}
    \circ \theta_{E_{K_E}, S_{K_S}} \circ \phi_{K_H, T} \\
    \theta_{e, s}(P_L, P_M) &= (P_L \arrowoplus s(e(P_M)), e(P_M)) \\
    \phi_{K_H, T}(L, R) &= (L, R \boxplus H_{K_H}(T, L)) \\
    \phi^{-1}_{K_H, T}(L, R) &= (L, R \boxminus H_{K_H}(T, L))
\end{align*}

These are families of length-preserving functions parameterized by the length
$|P_L| = |L| = |C_L| = l \in \mathbb{N}$;
for notational convenience we leave this parameter implicit.

We prove a security bound for HBSH in three stages:
\begin{itemize}
    \item we consider distinguishers for the inner construction $\theta$ in an
    attack model which forbids ``inner collisions'' in queries
    \item we prove a bound on the probability of an attacker causing an inner collision
    \item we put this together to bound the success probability of a distinguisher against
    HBSH.
\end{itemize}

At each stage, we consider an attacker \(\adv^{\calE, \calD}\) who makes $q$ queries
to oracles for
two length-preserving function families which take a tweak; the attacker is always free to vary
the length of input and tweak.

\begin{align*}
\calE, \calD:
\bin^* &\times \bin^l \times \bin^n \rightarrow \bin^l \times \bin^n \\
(C_L, C_R) &\leftarrow \calE_T(P_L, P_R) \\
(P_L, P_R) &\leftarrow \calD_T(C_L, C_R) \\
\end{align*}

\subsection{Inner part}\label{innerpart}

We consider an attacker trying to distinguish an idealized $\theta$ from a
pair of families of random length-preserving functions,
but we forbid the attacker from causing ``inner collisions'':
having made either of the queries: \label{constraints}
\begin{itemize}
    \item $(C_L, C_M) \leftarrow \calE_T(P_L, P_M)$
    \item $(P_L, P_M) \leftarrow \calD_T(C_L, C_M)$
\end{itemize}
both of the queries below are subsequently disallowed:
\begin{itemize}
    \item $\calE_\cdotp(\cdotp, P_M)$
    \item $\calD_\cdotp(\cdotp, C_M)$
\end{itemize}
where $\cdotp$ represents any value. However assuming $C_M \neq P_M$, this does not disallow
subsequent queries of the form $\calD_\cdotp(\cdotp, P_M)$ or $\calE_\cdotp(\cdotp, C_M)$
and it's important to consider such queries at each step below. We write
$P_M$/$C_M$ for the second argument and result to match notation used for $\theta$ elsewhere.
At this stage, we consider computationally unbounded attackers; we'll consider resource-bounded
attackers in~\autoref{composition}.

In what follows we use a standard concrete security hybrid argument per~\cite{concrete,games}:
for a fixed class of attacker, distinguishing advantage obeys the triangle inequality and
forms a pseudometric space. We consider a
sequence of experiments, bound the distinguishing advantage between successive
experiments, and thereby
prove an advantage bound for a distinguisher between the first and the last experiment
which is the sum of the advantage bound between each successive experiment.

\xprmtitle{innerpart}{randinner}: $\calE$ and $\calD$ are families of
random functions. Since by our constraints above every query to each
is distinct, every output of the
appropriate length is equally likely. Wherever we specify that an experiment uses multiple random
functions, those functions are independent unless stated otherwise.

\xprmtitle{innerpart}{notweak}: Ignore the tweak: let
$\barE, \barD$ be random length-preserving function families from
$\bin^l \times \bin^n \rightarrow \bin^l \times \bin^n$
and let $\calE_T = \barE$ and $\calD_T = \barD$ for all $T$.
Since by the above constraints, for each random function the second argument in every query
is still always distinct, every output of the appropriate length is equally
likely as before, and this is indistinguishable from \xprm{innerpart}{randinner}.

\xprmtitle{innerpart}{doublerf}: Use a Feistel network in which the stream cipher
nonce includes both $P_M$ and $C_M$.
\begin{itemize}
    \item $F_S: \bin^{2n} \rightarrow \bin^{l_S}$ is a random
    function
    \item $F_E, F_D: \bin^n \rightarrow \bin^n$ are random functions
    \item $\calE_T(P_L, P_M) = (P_L \arrowoplus F_S(P_M || F_E(P_M)), F_E(P_M))$
    \item $\calD_T(C_L, C_M) = (C_L \arrowoplus F_S(F_D(C_M) || C_M), F_D(C_M))$
\end{itemize}
Again, the constraints above ensure that for each random function,
every query is distinct,
every output of the appropriate length is equally
likely as before, and this is indistinguishable from
\xprm{innerpart}{notweak} and \xprm{innerpart}{randinner}.

\xprmtitle{innerpart}{rpswitch}: Substitute a random permutation for the pair of random functions.
\begin{itemize}
    \item $F_S: \bin^{2n} \rightarrow \bin^{l_S}$ is a random function as before
    \item $\pi: \bin^n \rightarrow \bin^n$ is a random permutation
    \item $\calE_T(P_L, P_M) = (P_L \arrowoplus F_S(P_M || \pi(P_M)), \pi(P_M))$
    \item $\calD_T(C_L, C_M) = (C_L \arrowoplus F_S(\pi^{-1}(C_M) || C_M), \pi^{-1}(C_M))$
    ie $\calD_T = \calE_T^{-1}$
\end{itemize}
The constraints above rule out ``pointless'' queries on $\pi$, so per section C of~\cite{cmc}
the advantage in distinguishing this from \xprm{innerpart}{doublerf} is at most
$2^{-n}\binom{q}{2}$.

\xprmtitle{innerpart}{halfrf}: Replace the two-argument $F_S$ with
a single-argument version which uses only $C_M$.
\begin{itemize}
    \item $F_S: \bin^n \rightarrow \bin^{l_S}$ is a random function
    \item $\pi: \bin^n \rightarrow \bin^n$ is a random permutation as before
    \item $\calE_T(P_L, P_M) = (P_L \arrowoplus F_S(\pi(P_M)), \pi(P_M))$
    ie $\calE_T = \theta_{\pi, F_S}$
    \item $\calD_T(C_L, C_M) = (C_L \arrowoplus F_S(C_M), \pi^{-1}(C_M))$
    ie $\calD_T = \theta_{\pi, F_S}^{-1}$
\end{itemize}
Since $\pi$ is a permutation,
for any pair of queries $C_M = C_M'$ if and only if $P_M = P_M'$.
This is therefore indistinguishable from \xprm{innerpart}{rpswitch}.
This is a small upside to HBSH's asymmetry: if a symmetrical
construction such as $F_S(P_M \oplus \pi(P_M))$ were used here instead, it would be
distinguishable with advantage $2^{-n}\binom{q}{2}$.

Summing these distances, with these constraints the advantage distinguishing between
\xprm{innerpart}{randinner} and \xprm{innerpart}{halfrf} is at most
$2^{-n}\binom{q}{2}$.

\subsection{Collision finding}\label{collision}
We now bound the probability that the attacker will cause an
``inner collision''---a query to the inner part which doesn't meet the constraints described
in \autoref{constraints}.

We assume $H$ is $\epsilon$-almost-$\Delta$-universal:
for any $g \in \bin^n$ and
any two distinct messages $(T, M) \neq (T', M')$,
$\probsub{K_H \sample \mathcal{K}_H}{H_{K_H}(T, M) \boxminus H_{K_H}(T', M') = g} \leq \epsilon$.
Adiantum and HPolyC differ only in the choice of $H$ function, and we defer the
description of these functions and their $\epsilon$A$\Delta$U property to \autoref{hashing},
noting here that for both, the value of $\epsilon$ will depend
on the maximum length of the message and tweak we allow the attacker to query for.

From here on, we forbid only ``pointless queries'': after either of the queries
$(C_L, C_R) \leftarrow \calE_T(P_L, P_R)$ or $(P_L, P_R) \leftarrow \calD_T(C_L, C_R)$,
both the subsequent queries
$\calE_T(P_L, P_R)$ and $\calD_T(C_L, C_R)$ would be forbidden. Again, we consider a
computationally unbounded attacker.

A hash key $K_H \sample \mathcal{K}_H$ is chosen at random, and for each query we define
\begin{itemize}
    \item $P_M = P_R \boxplus H_{K_H}(T, P_L)$
    \item $C_M = C_R \boxplus H_{K_H}(T, C_L)$
\end{itemize}

For query $1 \leq i \leq q$, we'll use superscripts to refer to the variables for that query, eg
$P_M^i$, $C_M^i$. The attacker wins if there exists $i < j \leq q$ such that
either
\begin{itemize}
    \item $j$ is a plaintext query (a query to $\calE$) and $P_M^i = P_M^j$ or
    \item $j$ is a ciphertext query (a query to $\calD$) and $C_M^i = C_M^j$.
\end{itemize}
We do not consider a query such that $C_M^i = P_M^j$ (or vice versa) a win.

Here we are considering not distinguishing probability but success probability; we
show a bound on success probability for the first experiment, and for each subsequent experiment
we bound the increase in success probability over the previous experiment.

\xprmtitle{collision}{keyend}: Choose responses fairly at random of the appropriate length;
once all $q$ queries are complete, choose the hash key and see if the attacker succeeded.

If query $j$ is a plaintext query, the attacker knows the query and result
for all $i < j$, and can choose plaintext values to maximize the probability of success.
If $T^i, P_L^i = T^j, P_L^j$ then the hashes will be the same, and since pointless
queries are forbidden we have that $P_R^i \neq P_R^j$ and therefore that
$P_M^i \neq P_M^j$. Otherwise by the $\epsilon$A$\Delta$U property,
$\prob{P_M^i = P_M^j} \leq \epsilon$. The same success bound holds for a ciphertext query.
The overall probability of success
is at most the sum of the probability of success for each pair:
$\epsilon\binom{q}{2}$.

Note that we don't assume that probabilities are independent here;
for any events $A$, $B$ we have that $\prob{A \vee B} \leq \prob{A} + \prob{B}$, with
equality if $A$, $B$ are disjoint.

\xprmtitle{collision}{keystart}: As with \xprm{collision}{keyend}, but
choose the hash key at the start of the experiment. This
does not change the probability of success.

\xprmtitle{collision}{earlystop}: As with \xprm{collision}{keystart}, but
end the experiment as soon as the attacker succeeds.
This doesn't change the success probability.

\xprmtitle{collision}{randomfuncs}: Use random function families for $\calE$,
$\calD$. Since we forbid pointless queries,
all responses of the appropriate length are equally likely as before,
and this doesn't change the success probability.

\xprmtitle{collision}{randinner}: Use random function families for $\calE'$, $\calD'$, and define
\begin{itemize}
    \item $\calE_T = \phi^{-1}_{K_H, T} \circ \calE_T' \circ \phi_{K_H, T}$
    \item $\calD_T = \phi^{-1}_{K_H, T} \circ \calD_T' \circ \phi_{K_H, T}$
\end{itemize}

Since a random function composed with a bijective function is a random function,
this doesn't change the success probability, which remains at most
$\epsilon\binom{q}{2}$.

\xprmtitle{collision}{halfrf}: The middle part of the sandwich is now a pair of random functions,
as per the first experiment in \autoref{innerpart}, \xprm{innerpart}{randinner}.
Substitute this with the last experiment, \xprm{innerpart}{halfrf}:

\begin{itemize}
    \item $F_S: \bin^n \rightarrow \bin^{l_S}$ is a random function
    \item $\pi: \bin^n \rightarrow \bin^n$ is a random permutation
    \item $\calE_T = \phi^{-1}_{K_H, T} \circ \theta_{\pi, F_S} \circ \phi_{K_H, T}$
    \item $\calD_T = \phi^{-1}_{K_H, T} \circ \theta_{\pi, F_S}^{-1} \circ \phi_{K_H, T} = \calE_T^{-1}$
\end{itemize}

Given any attacker against \xprm{collision}{halfrf},
we construct a distinguisher between \xprm{innerpart}{randinner} and \xprm{innerpart}{halfrf} as
follows: we choose a random $K_H$, and use our oracle for the inner part as $\theta$.
We report 1 if the attacker succeeds in generating an inner collision.
We stop the experiment early if the attacker succeeds, and any query in which the attacker
does not succeed is one that obeys the query bounds of \autoref{innerpart}.
Therefore, the difference in success probability
between \xprm{collision}{randinner} and \xprm{collision}{halfrf}
for these two outer experiments can be no more than
the advantage bound of $2^{-n}\binom{q}{2}$
established in~\autoref{innerpart} for distinguishing between
\xprm{innerpart}{randinner} and \xprm{innerpart}{halfrf}.
The success
probability for this final experiment is therefore at most
$(\epsilon + 2^{-n})\binom{q}{2}$.

\subsection{Composition}\label{composition}

Finally we put these pieces together to bound the advantage of distinguishing HBSH from
a family of random permutations. As before, the attacker can make encryption and decryption queries
but ``pointless'' queries are forbidden; in addition, the attacker is constrained by a time
bound $t$. We start with this experiment:

\xprmtitle{composition}{permutation}: For all lengths and all $T$, $\calE_T$ is a random permutation,
and $\calD_T = \calE_T^{-1}$. The security of a variable-length tweakable SPRP is defined by
the advantage bound in distinguishing it from this experiment.

\xprmtitle{composition}{randomfuncs}: $\calE$ and $\calD$ are
families of random functions.
Since pointless queries are forbidden, the advantage in distinguishing this from
\xprm{composition}{permutation} is at most $2^{-|P|}\binom{q}{2} \leq 2^{-n}\binom{q}{2}$
per section C of~\cite{cmc}.

\xprmtitle{composition}{randinner}: $\calE'$ and $\calD'$ are families of random functions.
Choose a hash key $K_H \sample \mathcal{K}_H$,
and conjugate the random functions by Feistel calls to the hash function:

\begin{itemize}
    \item $\calE_T = \phi^{-1}_{K_H, T} \circ \calE_T' \circ \phi_{K_H, T}$
    \item $\calD_T = \phi^{-1}_{K_H, T} \circ \calD_T' \circ \phi_{K_H, T}$
\end{itemize}

as per the step from \xprm{collision}{randomfuncs}
to \xprm{collision}{randinner}.
Since a random function composed with a bijective function is a random function,
this is indistinguishable from \xprm{composition}{randomfuncs}.

\xprmtitle{composition}{halfrf}: Substitute \xprm{innerpart}{halfrf} for \xprm{innerpart}{randinner}.
\begin{itemize}
    \item $K_H \sample \mathcal{K}_H$
    \item $F_S: \bin^n \rightarrow \bin^{l_S}$ is a random function
    \item $\pi: \bin^n \rightarrow \bin^n$ is a random permutation
    \item $\calE_T = \phi^{-1}_{K_H, T} \circ \theta_{\pi, F_S} \circ \phi_{K_H, T}$
    \item $\calD_T = \calE_T^{-1}$
\end{itemize}

By \autoref{collision} the attacker's probability of causing an inner collision is at most
$(\epsilon + 2^{-n})\binom{q}{2}$. If they do not cause a collision,
their queries to $\theta$ meet the constraints set out in \autoref{innerpart}, which
bounds the distinguishing advantage in this case to $2^{-n}\binom{q}{2}$.
The advantage in distinguishing from \xprm{composition}{randinner}
is at most the sum of the collision probability and the no-collision advantage, which is
$(\epsilon + 2(2^{-n}))\binom{q}{2}$.

\xprmtitle{composition}{block}: Choose a random $K_E \sample \mathcal{K}_E$ and
substitute a block cipher $E_{K_E}$ for $\pi$.
\begin{itemize}
    \item $K_H \sample \mathcal{K}_H$, $K_E \sample \mathcal{K}_E$
    \item $F_S: \bin^n \rightarrow \bin^{l_S}$ is a random function
    \item $\calE_T = \phi^{-1}_{K_H, T} \circ \theta_{E_{K_E}, F_S} \circ \phi_{K_H, T}$
    \item $\calD_T = \calE_T^{-1}$
\end{itemize}

By the standard argument
for such a substitution, the advantage for this substitution is at most
$\advantage{\pm \mathrm{prp}}{E_{K_E}}[(q, t')]$ where $t$ is a time bound on the attacker and
$t' = t + \bigO{\sum_i \left( |P^i| + |T^i| \right)}$.

\xprmtitle{composition}{xrf}: Use the random function $F_S$ to derive the keys
$K_E$, $K_H$:

\begin{itemize}
    \item $F_S: (\lambda \cup \bin^n) \rightarrow \bin^{l_S}$ is a random function
    \item $K_E || K_H || \ldots = F_S(\lambda)$
    \item $\calE_T = \phi^{-1}_{K_H, T} \circ \theta_{E_{K_E}, F_S} \circ \phi_{K_H, T}$
    \item $\calD_T = \calE_T^{-1}$
\end{itemize}

This is indistinguishable from \xprm{composition}{block}.

\xprmtitle{composition}{xchacha}: Choose a random $K_S \sample \mathcal{K}_S$ and
substitute a stream cipher $S_{K_S}$ for the random function $F_S$

\begin{itemize}
    \item $K_S \sample \mathcal{K}_S$
    \item $K_E || K_H || \ldots = S_{K_S}(\lambda)$
    \item $\calE_T = \phi^{-1}_{K_H, T} \circ \theta_{E_{K_E}, S_{K_S}} \circ \phi_{K_H, T}$
    \item $\calD_T = \calE_T^{-1}$
\end{itemize}

This is HBSH. Taking into account the $|K_E| + |K_H|$-bit
query to $S_{K_S}$ to derive $K_E$, $K_H$,
by the standard argument the advantage for this substitution is at most
$\advantage{\mathrm{prf}}{S_{K_S}}[\left(|K_E| + |K_H| + \sum_i \left(|P^i|-n\right), t'\right)]$
where the first argument measures not queries but bits of output.

Summing these,
the advantage in distinguishing HBSH from a family of random permutations is at most
\label{hbshadvantage}

\begin{align*}
& (\epsilon + 3(2^{-n}))\binom{q}{2} \\
+& \advantage{\pm \mathrm{prp}}{E_{K_E}}[(q, t')] \\
+& \advantage{\mathrm{prf}}{S_{K_S}}[\left(|K_E| + |K_H| + \sum_i \left(|P^i|-n\right), t'\right)]
\end{align*}
\end{document}
